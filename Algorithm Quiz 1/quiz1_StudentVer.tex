\documentclass[letterpaper, 11pt]{article}
\usepackage{latexsym}
\usepackage{amssymb}
\usepackage{times}
%\usepackage[in]{fullpage}
\usepackage{amsmath,amsfonts,amsthm}
\usepackage{graphicx}
\usepackage[]{algorithm2e}

\bibliographystyle{plain}

\begin{document}
\setlength{\parindent}{0em}
\setlength{\parskip}{1em}

\title{ Quiz \#1 \\ Introduction to Algorithms \\ 601.433/633 \\Spring 2020}
\date{}
\maketitle
\vspace{-5em}
You must submit your solutions by Wednesday, April 8th, 10am. Late submissions will NOT be accepted. You may submit handwritten answers -- you will have to scan/photograph them, convert it to a pdf and upload it to Gradescope.
%%%%%%%%%%%%%%%%%%%%%%%%%%%%%%%%%%

\section{ Problem 1 (50 points)}
For each statement below explain if it is true or false and in a couple of sentences provide an explanation for your answer. Be as mathematically precise as you can in your explanation. The base of $\log$ is $2$ unless stated otherwise.

\begin{enumerate}

	\item $2^{n^2} = \Theta(3^{n + \sqrt{n}})$

	
	\item  $n! = \omega(2^n)$
	

	\item  Let $f$ be positive function. Then $f(n) = O((f(n))^2)$.
	

\end{enumerate}


\newpage
\section{Problem 2 (50 points)}

Resolve the following recurrences in terms of a big-$\Theta$ bound. You may assume that $T(0) = T(1) = 1$. Provide a proof for the bound you give. If appropriate, you may invoke the Master Theorem (and the appropriate case).

\begin{enumerate}

\item $T(n) = T(n-1) + 2T(n-2)$


\item $T(n) = 7T(n/15) + n^5$

\end{enumerate}


\newpage

\section{Problem 3 (50 points)}
A sequence $a_1, a_2, \ldots, a_n$ has a special element if more than half of the elements
in the sequence are the same. For example, $3$ is a special element in the sequence
$7, 3, 3, 3, 1, 3, 3, 4, 5, 3$. On the other hand, the sequence $5, 4, 1, 1, 2, 3, 2, 3, 6$ has
no special element. Give a divide and conquer algorithm that runs in time
$O(n \log n)$ and returns a special element in a sequence of $n$ numbers
or returns $None$ if no such element exists. Prove the correctness of your algorithm
and prove that its runtime is $O(n\log n)$. (Note: there exists an $O(n)$ algorithm to
solve this problem that doesn’t make use of divide and conquer– if you figure it
out, you may prove its correctness and runtime instead.)





\newpage

\section{Problem 4 (50 points)}
You are given $n$ tasks for a machine. Task $i$ is described by $l_i = [a_i, b_i]$ on the real line, where $a_i, b_i$ are real numbers, $a_i \leq b_i$ and $1 \leq i \leq n$. Give an algorithm that computes the total times of this set of tasks,
that is, the length of $\cup_{i=1}^{n}l_i$ in $O(n\log n)$ time.


For example, for the set of tasks $\{[1,3], [2 ,4.5], [6, 9], [7,8]\}$, the total time is $(4.5 - 1) + (9-6) = 6.5$.

Make sure to prove the correctness and running time of your algorithm. 





\newpage
\section{Problem 5 (50 points)}

Suppose that you have a set of $n$ integers, $A = \{a_1, . . . , a_n\}$, each of them is between $0$ and $K$ (inclusive). Your goal is to find
a partition of $A$ into two sets $S_1$ and $S_2$ (so $S_1\cup S_2 = A$ and $S_1\cap S_2 = \emptyset$)
that minimizes $|W(S_1)-W(S_2)|$ where $W(S)$ denote the sum of integers in $S$. Your algorithm's running time should be polynomial in $n$ and $K$.

Make sure to prove the correctness (mainly the optimal substructure property) and running time. 


\end{document}

%%%%%%%%%%%%%%%%%%%%%%%%%%%%%
\grid
