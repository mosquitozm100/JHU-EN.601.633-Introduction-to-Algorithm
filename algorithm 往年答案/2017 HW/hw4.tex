\documentclass[letterpaper, 11pt]{article}
\usepackage{latexsym}
\usepackage{amssymb}
\usepackage{times}
%\usepackage[in]{fullpage}
\usepackage{amsmath,amsfonts,amsthm}
\usepackage{graphicx}

%\documentclass[11pt]{article}
%\pagestyle{myheadings}
%\usepackage[ruled,nothing]{algorithm}
%\usepackage{algorithmic}
%\usepackage[dvips]{epsfig,graphicx}
%\numberwithin{equation}{section}

\bibliographystyle{plain}

\newenvironment{newalgo}[2]{\begin{algorithm}

\caption{\textsc{#1}}\label{#2}

\begin{algorithmic}[1]}{\end{algorithmic}\end{algorithm}}



\newcommand{\gm}{\gamma}
\newcommand{\wh}{\widehat}
\newcommand{\rep}{representation}
\newcommand{\rv}{random variable}
\newcommand{\la}{\lambda}
\newcommand{\wt}{\widetilde}
\newcommand{\st}{such that}
\newcommand{\slvary}{slowly varying}
\newcommand{\ma}{moving average}
\newcommand{\regvary}{regularly varying}
\newcommand{\asy}{asymptotic}
\newcommand{\ts}{time series}
\newcommand{\id}{infinitely divisible}
\newcommand{\seq}{sequence}
\newcommand{\fidi}{finite dimensional \ds}

\newcommand{\ble}{\begin{lemma}}
\newcommand{\ele}{\end{lemma}}
\newcommand{\bfX}{{\bf X}}
\newcommand{\pro}{probabilit}
\newcommand{\BX}{{\bf X}}
\newcommand{\BY}{{\bf Y}}
\newcommand{\BZ}{{\bf Z}}
\newcommand{\BV}{{\bf V}}
\newcommand{\BW}{{\bf W}}
\newcommand{\reals}{{\mathbb R}}
\newcommand{\bbr}{\reals}

\newcommand{\balpha}{\mbox{\boldmath$\alpha$}}
\newcommand{\bbeta}{\mbox{\boldmath$\beta$}}
\newcommand{\bmu}{\mbox{\boldmath$\mu$}}
\newcommand{\tbmu}{\mbox{\boldmath${\tilde \mu}$}}
\newcommand{\bEta}{\mbox{\boldmath$\eta$}}


\def \br#1{\left \{#1 \right \}}
\def \pr#1{\left (#1 \right)}

\newcommand{\Gm}{\Gamma}
\newcommand{\ep}{\epsilon}


\newtheorem{lemma}{Lemma}[section]
\newtheorem{figur}[lemma]{Figure}
\newtheorem{theorem}[lemma]{Theorem}
\newtheorem{proposition}[lemma]{Proposition}
\newtheorem{definition}[lemma]{Definition}
\newtheorem{corollary}[lemma]{Corollary}
\newtheorem{example}[lemma]{Example}
\newtheorem{exercise}[lemma]{Exercise}
\newtheorem{remark}[lemma]{Remark}
\newtheorem{fig}[lemma]{Figure}
\newtheorem{tab}[lemma]{Table}
\newtheorem{fact}[lemma]{Fact}
\newtheorem{test}{Lemma}
\newtheorem{algorithm}[lemma]{Algorithm}

\newcommand{\play}{\displaystyle}

\newcommand{\ms}{measure}
\newcommand{\beao}{\begin{eqnarray*}}
\newcommand{\eeao}{\end{eqnarray*}\noindent}
\newcommand{\beam}{\begin{eqnarray}}
\newcommand{\eeam}{\end{eqnarray}\noindent}

\newcommand{\halmos}{\hfill\mbox{\qed}\\}
\newcommand{\fct}{function}
\newcommand{\ins}{insurance}
\newcommand{\ds}{distribution}

\newcommand{\one}{{\bf 1}}
\newcommand{\eid}{\buildrel{\rm d}\over {=}}
\newcommand {\Or}{\rm ORDER}
\newcommand {\In}{\rm INTER}

\newcommand{\bbd}{{\mathbb D}}
\newcommand{\vi}{$V_{ij}$ }
\newcommand{\rr}{R^{\prime\prime}}
%\newcommand{\R}{R^\prime}
\newcommand{\ci}{\frac{1}{c}}
\newcommand{\Vi}{V(n)}
\newcommand{\dR}{\mathcal R}
\newcommand{\md}[1]{\left(\ \rm{mod}\ \it{#1}\right)}
\newcommand{\So}{s}
%\begin{document}
%\def\DoubleSpace{\baselineskip=24pt}
%\DoubleSpace \sloppy

\begin{document}



\title{Homework \#4 \\ Introduction to Algorithms/Algorithms 1 \\ 600.363/463 \\Spring 2017}
\author{\textbf{Due on:} Tuesday, March 7th, 11:59pm \\
\textbf{Late submissions:} will NOT be accepted\\
\textbf{Format:} Please start each problem on a new page.
\\\textbf{Where to submit:} On Gradescope, under HW4
\\ Please type your answers; handwritten assignments will not be accepted.
\\ To get full credit, your answers must be explained clearly,\\ with enough details
and rigorous proofs.
\\}

\maketitle

%%%%%%%%%%%%%%%%%%%%%%%%%%%%%%%%%%



\section{Problem 1 (20 points)} % #6
When you are checking out at BalMart,  you want to make change for $A$ cents.  Assuming that the cashier has infinite supply of each of $C= \{ C_1, C_2,\dots,C_t\}$ valued coins, can you count how many ways to make change for $A$ cents? Give an efficient dynamic programming algorithm and analyze the running time. Here we don't consider the order of the coins.

For example, when $A = 3$ and $C = \{1,2,3\}$, there are three solutions: $\{1,1,1\}$, $\{1,2\}$, $\{3\}$. 

\section{Problem 2 (20 points)}
Suppose you are managing the construction of billboards on an east-west highway that extends in a straight line. The possible sites for billboards are given by numbers $x_1, x_2,\dots, x_n$ with $0 \le x_1 < x_2 < \dots < x_n$, specifying their distance in miles from the west end of the highway. If you place a billboard at location $x_i$, you receive payment $p_i > 0$.\\\\
Regulations imposed by the Baltimore County's Highway Department require that any pair of billboards be more than 5 miles apart. You'd like to place billboards at a subset of the sites so as to maximize your total revenue, subject to that placement restriction.\\
 For example, suppose n = 4, with\\
 \begin{align*}
 <x_1,x_2,x_3,x_4> = <6,7,12,14>,
 \end{align*}
and 
\begin{align*}
<p_1,p_2,p_3,p_4> = <5,6,5,1>.
\end{align*}

The optimal solution would be to place billboards at $x_1$ and $x_3$, for a total revenue of $p_1 +p_3 = \$10$.

Give an efficient dynamic-programming algorithm that takes as input an instance (locations $\{x_i\}$ given in sorted order and their prices $\{p_i\}$) and returns the maximum revenue obtainable from a valid subset of sites. Analyze the running time of your algorithm. Your solution must clearly define a recursive formula.


\end{document}

%%%%%%%%%%%%%%%%%%%%%%%%%%%%%
