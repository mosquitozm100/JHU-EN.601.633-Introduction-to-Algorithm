\documentclass[letterpaper, 11pt]{article}
\usepackage{latexsym}
\usepackage{amssymb}
\usepackage{times}
%\usepackage[in]{fullpage}
\usepackage{amsmath,amsfonts,amsthm}
\usepackage{graphicx}

%\documentclass[11pt]{article}
%\pagestyle{myheadings}
%\usepackage[ruled,nothing]{algorithm}
%\usepackage{algorithmic}
%\usepackage[dvips]{epsfig,graphicx}
%\numberwithin{equation}{section}

\bibliographystyle{plain}

\newenvironment{newalgo}[2]{\begin{algorithm}

\caption{\textsc{#1}}\label{#2}

\begin{algorithmic}[1]}{\end{algorithmic}\end{algorithm}}



\newcommand{\gm}{\gamma}
\newcommand{\wh}{\widehat}
\newcommand{\rep}{representation}
\newcommand{\rv}{random variable}
\newcommand{\la}{\lambda}
\newcommand{\wt}{\widetilde}
\newcommand{\st}{such that}
\newcommand{\slvary}{slowly varying}
\newcommand{\ma}{moving average}
\newcommand{\regvary}{regularly varying}
\newcommand{\asy}{asymptotic}
\newcommand{\ts}{time series}
\newcommand{\id}{infinitely divisible}
\newcommand{\seq}{sequence}
\newcommand{\fidi}{finite dimensional \ds}

\newcommand{\ble}{\begin{lemma}}
\newcommand{\ele}{\end{lemma}}
\newcommand{\bfX}{{\bf X}}
\newcommand{\pro}{probabilit}
\newcommand{\BX}{{\bf X}}
\newcommand{\BY}{{\bf Y}}
\newcommand{\BZ}{{\bf Z}}
\newcommand{\BV}{{\bf V}}
\newcommand{\BW}{{\bf W}}
\newcommand{\reals}{{\mathbb R}}
\newcommand{\bbr}{\reals}

\newcommand{\balpha}{\mbox{\boldmath$\alpha$}}
\newcommand{\bbeta}{\mbox{\boldmath$\beta$}}
\newcommand{\bmu}{\mbox{\boldmath$\mu$}}
\newcommand{\tbmu}{\mbox{\boldmath${\tilde \mu}$}}
\newcommand{\bEta}{\mbox{\boldmath$\eta$}}


\def \br#1{\left \{#1 \right \}}
\def \pr#1{\left (#1 \right)}

\newcommand{\Gm}{\Gamma}
\newcommand{\ep}{\epsilon}


\newtheorem{lemma}{Lemma}[section]
\newtheorem{figur}[lemma]{Figure}
\newtheorem{theorem}[lemma]{Theorem}
\newtheorem{proposition}[lemma]{Proposition}
\newtheorem{definition}[lemma]{Definition}
\newtheorem{corollary}[lemma]{Corollary}
\newtheorem{example}[lemma]{Example}
\newtheorem{exercise}[lemma]{Exercise}
\newtheorem{remark}[lemma]{Remark}
\newtheorem{fig}[lemma]{Figure}
\newtheorem{tab}[lemma]{Table}
\newtheorem{fact}[lemma]{Fact}
\newtheorem{test}{Lemma}
\newtheorem{algorithm}[lemma]{Algorithm}

\newcommand{\play}{\displaystyle}

\newcommand{\ms}{measure}
\newcommand{\beao}{\begin{eqnarray*}}
\newcommand{\eeao}{\end{eqnarray*}\noindent}
\newcommand{\beam}{\begin{eqnarray}}
\newcommand{\eeam}{\end{eqnarray}\noindent}

\newcommand{\halmos}{\hfill\mbox{\qed}\\}
\newcommand{\fct}{function}
\newcommand{\ins}{insurance}
\newcommand{\ds}{distribution}

\newcommand{\one}{{\bf 1}}
\newcommand{\eid}{\buildrel{\rm d}\over {=}}
\newcommand {\Or}{\rm ORDER}
\newcommand {\In}{\rm INTER}

\newcommand{\bbd}{{\mathbb D}}
\newcommand{\vi}{$V_{ij}$ }
\newcommand{\rr}{R^{\prime\prime}}
%\newcommand{\R}{R^\prime}
\newcommand{\ci}{\frac{1}{c}}
\newcommand{\Vi}{V(n)}
\newcommand{\dR}{\mathcal R}
\newcommand{\md}[1]{\left(\ \rm{mod}\ \it{#1}\right)}
\newcommand{\So}{s}
%\begin{document}
%\def\DoubleSpace{\baselineskip=24pt}
%\DoubleSpace \sloppy

\begin{document}

\title{Homework \#1 \\ Introduction to Algorithms/Algorithms 1 \\ 600.363/463 \\ Spring 2017}
\author{\textbf{Due on:} Tuesday, February 7th, 5pm \\
\textbf{Late submissions:} will NOT be accepted\\
\textbf{Format:} Please start each problem on a new page.
\\\textbf{Where to submit:} On blackboard, under student assessment. \\
Otherwise, please bring your solutions to the lecture.
\\}


\maketitle



\section{Problem 1 (20 points)}

\subsection{(5 points)}

Prove that, if $\lim_{n \to \infty} \frac{f(n)}{g(n)} = k$, where $k$ is a positive constant, then $f(n) = \Theta(g(n))$.

\subsection{(5 points)}
For each statement below explain if it is true or false and prove
your answer. Be as precise as you can. The base of $\log$ is $2$ unless stated otherwise.

\begin{enumerate}

\item  ${n^2\over \log n} = \Theta(n)$

\item $2^n = O(3^{n})$

\item $\sqrt{n} = \Theta(2\log n^2)$

\item $3n\log n + n = O(\frac{n^2-n}{2})$

\item Let $f$ and $g$ be positive functions. If $f(n)+g(n) = \Omega(f(n))$ then $g(n) = O((f(n))^2)$.

\end{enumerate}

\subsection{(10 points)}

\begin{enumerate}

\item Prove that $$\sum_{i=1}^n {\log i} = O(n \log n).$$

\end{enumerate}

\section{Problem 2 (20 Points)}

\subsection{(10 points)}

\begin{enumerate}
\item Prove by induction that $\sum_{i=1}^n \frac{1}{i(i+1)} = \frac{n}{n+1}$ for all $n\ge 1$.

\item Alice wants to distribute three movie tickets to ten friends. If each friend could get up to one ticket, how many ways can Alice distribute these tickets?

\item We have $n$ books. Each book, independently and randomly, is placed into one of $n$ shelves. What is the probability that there are no empty shelves at the end of our experiment?
\end{enumerate}

\subsection{(10 points)}

You are given $n=2^k$ compact discs, all of which look identical. However, there is one defective disc among all discs --- it weighs different than the rest. You are also given an equivalence tester, which has two compartments. You could place any set of objects in each compartment, and the
tester tells you whether or not the two sets weigh the same. Note that the tester doesn't tell you which side is heavier and which one lighter. \\\\
Your goal is to use the tester at most $k$ times to determine which of the discs is defective. You may assume that $k > 1$. Can you describe your method and briefly explain why it works? (Note that you are allowed to use the tester only $k$ times, not $O(k)$. You will receive partial credit for slightly worse solutions.)


\end{document}



































