\documentclass[letterpaper, 11pt]{article}
\usepackage{latexsym}
\usepackage{amssymb}
\usepackage{times}
%\usepackage[in]{fullpage}
\usepackage{amsmath,amsfonts,amsthm}
\usepackage{graphicx}
\usepackage{color}
\usepackage{xcolor}
\usepackage{algorithm}
\usepackage{algpseudocode}

%\documentclass[11pt]{article}
%\pagestyle{myheadings}
%\usepackage[ruled,nothing]{algorithm}
%\usepackage{algorithmic}
%\usepackage[dvips]{epsfig,graphicx}
%\numberwithin{equation}{section}

\bibliographystyle{plain}


\begin{document}
\setlength{\parskip}{1em}
\setlength{\parindent}{0em}


\title{Homework \#4 \\ Introduction to Algorithms/Algorithms 1 \\ 601.433/633 \\Spring 2020}
\author{\textbf{Due on:} Tuesday, March 10th, 12pm \\
\\\textbf{Where to submit:} On Gradescope.
\\ Please type your answers; handwritten assignments will not be accepted.}
\date{}

\maketitle

%%%%%%%%%%%%%%%%%%%%%%%%%%%%%%%%%%

\section{Problem 1 (15 points)}
Suppose you are managing the construction of billboards on an east-west highway that extends in a straight line. The possible sites for billboards are given by reals $x_1, x_2,\dots, x_n$ with $0 \le x_1 < x_2 < \dots < x_n$, specifying their distance in miles from the west end of the highway. If you place a billboard at location $x_i$, you receive payment $p_i > 0$.\\\\
Regulations imposed by the Baltimore County's Highway Department require that any pair of billboards be more than 5 miles apart. You'd like to place billboards at a subset of the sites so as to maximize your total revenue, subject to that placement restriction.

For example, suppose $n$ = 4, with
\begin{align*}
 \langle x_1,x_2,x_3,x_4 \rangle = \langle 6,7,12,14 \rangle,
\end{align*}
and
\begin{align*}
\langle p_1,p_2,p_3,p_4\rangle = \langle 5,6,5,1 \rangle.
\end{align*}

The optimal solution would be to place billboards at $x_1$ and $x_3$, for a total revenue of $p_1 +p_3 = \$10$.

Give an $O(n)$ time dynamic-programming algorithm that takes as input an instance (locations $\{x_i\}$ given in sorted order and their prices $\{p_i\}$) and returns the maximum revenue obtainable. As usual, prove correctness and running time of your algorithm.

\textcolor{red}{EDIT: We will give full credit for an $O(n\log(n))$ algorithm too.}

% \begin{proof}
% %Write your proof here.
% \end{proof}





\section{Problem 2 (20 points)}
You are given two numbers, $n$ and $k$, such that $n\in \mathbb{N}$ and $k\in \left\{1,\dots,9 \right\}$. Use dynamic programming to devise an algorithm which will find the number of $2n$-digit integers for which the sum of the first $n$ digits is equal to the sum of the last $n$ digits and each digit takes a value from $0$ to $k$.

For example, when $k = 2$ and $n = 1$: you have only $3$ such numbers $00$, $11$, $22$. 
For example, when $k = 1$ and $n = 2$: you have only $6$ such numbers $0000$, $0101$, $0110$, $1001$ , $1010$, $1111$.		

Your algorithm should work in time polynomial of $n$ and $k$. Prove correctness and provide running time analysis.


% \begin{proof} 
% %Write your proof here.
% \end{proof}





\section{Problem 3 (15 points)}

Alice and Bob found a treasure chest with different golden coins, jewelry and various old and expensive goods.
After evaluating the price of each object they created a list $P = \{p_1,\ldots,p_n\}$ for all $n$ objects, where $p_i \in \{1,\ldots, K\}$
is the price of the object $i$. Help Alice and Bob to check if the treasure can be divided  equally,
i.e. if it is possible to  break the set of all abjects $P$
into two parts $P_A$ and $P_B$ such that  $P_A \cup P_B = P$, $P_A \cap P_B = \emptyset$ and $\sum_{i\in P_A}{p_i} =  \sum_{i \in P_B}{p_i}$?

Your algorithm should run in time polynomial in $n$ and $K$. As usual, prove correctness and running time of your algorithm.


% \begin{proof} 
% %Write your proof here.
% \end{proof}

\end{document}


%%%%%%%%%%%%%%%%%%%%%%%%%%%%%
