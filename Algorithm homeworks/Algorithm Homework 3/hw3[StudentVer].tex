\documentclass[letterpaper, 11pt]{article}
\usepackage{latexsym}
\usepackage{amssymb}
\usepackage{times}
%\usepackage[in]{fullpage}
\usepackage{amsmath,amsfonts,amsthm}
\usepackage{graphicx}

%\documentclass[11pt]{article}
%\pagestyle{myheadings}
\usepackage[ruled,nothing]{algorithm}
\usepackage{algorithmic}
%\usepackage[dvips]{epsfig,graphicx}
%\numberwithin{equation}{section}

\bibliographystyle{plain}

\setlength{\parindent}{0em}
\setlength{\parskip}{1em}


\begin{document}

\title{Homework \#3 \\ Introduction to Algorithms \\ 601.433/633 \\ Spring 2020}
\author{\textbf{Due on:} Tuesday, February 25th, 12pm \\
% \textbf{Late submissions} will NOT be accepted\\
\textbf{Format:} Please start each problem on a new page.
\\\textbf{Where to submit:} On Gradescope, please mark the pages for each question
\\}
\date{}
\vspace{-0.5cm}
\maketitle

%%%%%%%%%%%%%%%%%%%%%%%%%%%%%%%%%%
\section{Problem 1 (24 points)}
Recall that when using the QuickSort algorithm to sort an array $A$ of length $n$, we picked an element $x \in A$ which we called the \emph{pivot} and split the array $A$ into two arrays $A_S, A_L$ such that $\forall y \in A_S, y \leq x$ and $\forall y \in A_L, y > x$. 

We will say that a pivot from an array $A$ provides $t|n-t$ separation if $t$ elements in $A$ are smaller than or equal to the pivot, and $n-t$ elements are strictly larger than the pivot.

Suppose Bob knows a secret way to find a good pivot with $\frac{n}{3} | \frac{2n}{3}$ separation in constant time. But at the same time Alice knows her own secret technique, which provides separation $\frac{n}{4} | \frac{3n}{4}$, her technique also works in constant time.

Recall that in the QuickSort algorithm, as per Section 7.1 in CLRS, the PARTITION subroutine picks a pivot $x$ from $A$ by simply picking the first element in the array. Alice and Bob's subroutines are subroutines for picking the pivot $x$ in the PARTITION subroutine for QuickSort. 

Alice and Bob applied their secret techniques as subroutines in the QuickSort algorithm to pick pivots. Whose algorithm works \textbf{asymptotically} faster? Or are the runtimes \textbf{asymptotically} the same? Prove your statement.



% \begin{proof}
% Write your proof here.
% \end{proof}



\section{Problem 2 (13 points)}
Resolve the \textbf{asymptotic complexity} of the following recurrences, i.e., solve them and give your answer in Big-$\Theta$ notation. Use Master theorem, if applicable.
In all examples assume that $T(1) = 1.$
To simplify your analysis, you can assume that $n = a^k$ for some $a, k$.

Your final answer should be as simple as possible, i.e., it should not contain any sums, recurrences, etc. 

\begin{enumerate}
\item $T(n) = 2T(n/8) + n^{\frac{1}{5}}\log n \log \log n $
% \begin{proof}
% Write your proof here.
% \end{proof}

\item $T(n) = 8T(n/2) + n^3 - 8n\log n$

% \begin{proof}
% Write your proof here.
% \end{proof}

\item $T(n) = T(n/2) + \log n$

% \begin{proof}
% Write your proof here.
% \end{proof}

\item $T(n) = T(n-1) + T(n-2)$

% \begin{proof}
% Write your proof here.
% \end{proof}



\item $T(n) = 3T(n^{\frac{2}{3}}) + \log n$

% \begin{proof}
% Write your proof here.
% \end{proof}

\end{enumerate}




\section{Problem 3 (13 points)}  Let $A$ and $B$ be two sorted arrays of $n$ elements each. We can easily find the median element in $A$ -- it is just the element in the middle -- and similarly we can easily find the median element in $B$. (Let us define the median of $2k$ elements as the element that is greater than $k-1$ elements and less than $k$ elements.) However, suppose we want to find the median element overall -- i.e., the $n$th smallest in the union of $A$ and $B$. 

Give an $O(\log n)$ time algorithm to compute the median of $A \cup B$. You may assume there are no duplicate elements.

As usual, prove correctness and the runtime of your algorithm. 


% \begin{proof}
% Write your proof here.
% \end{proof}


\end{document}

%%%%%%%%%%%%%%%%%%%%%%%%%%%%%
