\documentclass[letterpaper, 11pt]{article}
\usepackage{latexsym}
\usepackage{amssymb}
\usepackage{times}
%\usepackage[in]{fullpage}
\usepackage{amsmath,amsfonts,amsthm}
\usepackage{graphicx}
\usepackage{xcolor}

%\documentclass[11pt]{article}
%\pagestyle{myheadings}

\usepackage[ruled,nothing]{algorithm}
\usepackage{algorithmic}

\begin{document}



\title{Homework \#2 \\ Introduction to Algorithms \\ 601.433/633 \\ Spring 2020}
\author{\textbf{Due on:} Tuesday, February 13th, 12pm \\
\textbf{Format:} Please start each problem on a new page.
\\\textbf{Where to submit:} On Gradescope, please mark the pages for each question
\\}


\maketitle

%%%%%%%%%%%%%%%%%%%%%%%%%%%%%%%%%%

\section{Problem 1 (12 points)} % #6
Given a list of $n$ integers $x_1, \dots, x_n$ (possibly negative), find the indices $i, j \in [n]$ ($i \neq j$) such that $x_i \cdot x_j$ is maximized. Your algorithm must run in $O(n)$ time. 

% \begin{proof}
% Write your proof here. 
% \end{proof}


\section{Problem 2 (12 points)}
Let $S$ be an array of integers $\{S[1], S[2],\dots,S[n]\}$ such that $S[1] < S[2] < \dots < S[n]$. Design an algorithm to determine whether there exists an index $i$ such at $S[i] = i.$ For
example, in $\{-1,2\}$, $S[2]= 2$.

Your algorithm should work in $O(\log n)$ time. Prove the correctness of your algorithm.

% \begin{proof}
% Write your proof here. 
% \end{proof}

\section{Problem 3 (13 points) }
We say a 3-tuple of positive real numbers $(x_1, x_2, x_3)$ is \emph{legal} if a triangle can have sides of length $x_1, x_2$ and $x_3$. 
Given a list of $n$ positive real numbers $\{x_1, \dots, x_n\}$, count the number of unordered 3-tuples $(x_i, x_j, x_k)$ that are legal. For example, for the numbers $\{3,5,8,4,4\}$, $(3,4,5)$ is a legal tuple while $(4,4,8)$ is not.

Your algorithm should run in $O(n^2)$ time. Prove correctness of your algorithm. 

\textcolor{red}{EDIT: You may give an $O(n^2 \log n)$ time algorithm and get full-credit.}

% \begin{proof}
% Write your proof here. 
% \end{proof}

\section{Problem 4 (13 points)}
You are given one unsorted integer array $A$ of size $n$. You know that $A$ is almost sorted, that is it contains at most $m$ inversions, where inversion is a pair of indices $(i,j)$ such that $i < j$ and $A[i] > A[j]$.  
\begin{enumerate}

\item To sort array $A$ you applied algorithm Insertion Sort. Prove that it will take at most $O(n + m)$ steps. 

% \begin{proof}
% Write your proof here. 
% \end{proof}

\item What is a maximum possible number of inversions in the integer array of size $n$?       
% \begin{proof}
% Write your proof here. 
% \end{proof}

\end{enumerate}


\end{document}

%%%%%%%%%%%%%%%%%%%%%%%%%%%%%
